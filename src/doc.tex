\documentclass[11pt]{article}
\usepackage{times}
\usepackage{pldoc}
\sloppy
\makeindex

\begin{document}
% This LaTeX document was generated using the LaTeX backend of PlDoc,
% The SWI-Prolog documentation system



\section{lazy_streams.pl}

\label{sec:lazystreams}

\begin{description}
    \predicate{ask}{2}{+Generator, -NextValue}
the \arg{Generator} generator protocol works as follows:
A generator step is a call to a closure that moves its state forward
defining a generator simply stores it as a Prolog fact.

The predicate \predref{ask}{2} queries a generator if it is not done and
it marks a generator as "done" after its first failure.
This ensures it can be garbage collected
by making its handle unreacheable.

\predref{ask}{2} extracts X by calling state transformer E.

    \predicate{stop}{1}{+Generator}
\predref{stop}{1} marks a generator as done.
Future calls to it will fail

    \predicate{is_done}{1}{+Generator}
checks if a generator is done

    \predicate{empty}{1}{-Done}
empty stream generator, set to "done" up front

    \infixop{in}{-Value}{+Generator}
\predref{in}{2} backtracks over progressively advancing states.
\predref{in}{2} is an xfx infix operator of priority 800

    \predicate{show}{2}{+NumberOfItems, +Generator}
\predref{show}{2} collects results after K steps and prints them out
same as: \verb$show(K,Stream)$\Sneck{}\verb$once(findnsols(K,X,X in Stream,Xs))$,\verb$writeln(Xs)$.

    \predicate{show}{1}{+Generator}
collects and prints 12 results of \arg{Generator}

    \predicate{const}{2}{+Constant, -Generator}
Builds a constant infinite stream returning its first argument.
the "next" step, \verb$call(=(C),X)$ will simply unify X and C

    \predicate{rand}{1}{+RandomStreamGenerator}
produces a stream of random floating point numbers between 0 and 1

    \predicate{nat}{1}{-NaturalNumberStream}
Natural number generator, storing the next and its initial state.

    \predicate{pos}{1}{-PositiveIntegerStream}
stricly positive integers

    \predicate{neg}{1}{-NegativeIntgerStream}
strictly negative integers

    \predicate{list}{2}{+ListOrLazyList, -Stream}
Builds stream generator from list or lazy list. 

    \predicate{range}{3}{+From, +To, -RangeStream}
finite positive integer range generator

    \predicate{cycle}{2}{+StreamGenerator, -CycleStreamGenerator}
transforms a finite generator into an infinite cycle
advancing over its elements repeatedly.
Uses a circular list, unified with its own tail.

    \predicate{eng}{3}{+AnswerTemplate, +Goal, -Generator}
\arg{Generator} exposing the work of an engine as a stream of answers. 

    \predicate{ceng}{3}{+Answertemplate, +Goal, -Generator}
Clonable generator exposing the work of an engine as a stream of answers.
It works on a generator wrapping an engine
such that its goal and answer template are kept.
That makes it clonable, assuming it runs code that's side-effect free.

    \predicate{ceng_clone}{2}{+Generator, -Clone}
creates new generator from a generator's goal

    \predicate{take}{3}{+K, +Generator, -NewGenerator}
Builds generator for initial segment of length \arg{K} of given generator. 

    \predicate{drop}{3}{+K, +Generator, -NewGenerator}
Roll the stream to first postion after first \arg{K} items.
Returns generator positioned \arg{K} steps forward.

    \predicate{slice}{4}{+From, +To, +Generator, -NewGenerator}
Builds generator for a slice of a given stream \arg{From}..\arg{To} (excluding \arg{To}).

    \predicate{setify}{2}{+Gen, -NewGen}
Transforms a generator into one that produces distinct elements.
It avoids sorting and uses the built-in \predref{distinct}{2} to ensure
that it also works on infinite streams.

    \predicate{map}{3}{+Closure, +Generator, -NewGenerator}
Builds a generator that will apply a closure to each element of a given generator.

    \predicate{map}{4}{+Closure, +Generator1, +Generator2, -NewGenerator}
Builds a generator that combines two gnerators by creating
an advancer that applies a \arg{Closure} to their "next" yields.

    \predicate{reduce}{4}{+Closure, +Generator, +InitialVal, -ResultGenerator}
Builds generator that reduces given generator's yields with given closure,
starting with an initial value. Yields the resulting single final value.

    \predicate{do}{1}{+Goal}
Bactracks over \arg{Goal} for its side-effects only. 

    \predicate{zipper_of}{3}{+Generator1, +Generator2, -NewGenerator}
\predref{zipper_of}{3} collects pairs of elements in matching positions
in given two generators, finite of the same length or infinite.

    \predicate{arith_sum}{3}{+Gen1, +Gen2, -NewGen}
Elementwise addition of two streams.

    \predicate{mult}{3}{+X, +Y, -P}
\arg{P} is the result of the multiplication of two numbers

    \predicate{arith_mult}{3}{+Gen1, +Gen2, -NewGen}
Elementwise multiplication of two streams.

    \predicate{chain}{3}{+Closure, +Generator, -Newgenerator}
Pipes elements of a stream through a transformer.

    \predicate{chains}{3}{+ListOfClosures, +Generator, -Newgenerator}
Pipes stream through a list of transformers.

    \predicate{mplex}{3}{+Closures, +Gen, -Values}
multiplexes a stream through a list of transfomers
returns the list of values obtained by appying each
transformer to the next lement of the generator

    \predicate{clause_stream}{2}{+Head, -StreamOfMatchingClauses}
generates a stream of clauses matching a given goal

    \predicate{sum}{3}{+Gen1, +Gen2, -NewGen}
Interleaved sum merging two finite or infinite generators.

    \predicate{cat}{2}{+GeneratorList, -ConcatenationOfGenerators}
concatenates streams of a list of generators
Int only makes sense if all but the last one are finite.

    \predicate{prod_}{3}{+Gen1, +Gen2, -NewGen}
engine-based direct product

    \predicate{prod}{3}{+Gen1, +Gen2, -NewGen}
direct product of two finite or infinite generators

    \predicate{cantor_pair}{3}{+Int1, +Int2, -Int}
Cantor's pairing function

    \predicate{cantor_unpair}{3}{+Int, -Int1, -Int2}
Inverse of Cantor's pairing function.

    \predicate{int_sqrt}{2}{+PosInt, -IntSqrt}
computes integer square root using Newton's method

    \predicate{conv}{1}{-Generator}
\arg{Generator} for N * N self-convolution.

    \predicate{lazy_nats}{1}{-LazyListOfNaturalNumbers}
infinite lazy list of natural numbers

    \predicate{gen2lazy}{2}{+Generator, -LazyLIst}
Turns a generator into a lazy list 

    \predicate{lazy2gen}{2}{+LazyList, -Generator}
Turns a lazy list into a generator.
Note that \predref{list}{2} actually just works on lazy lists!

    \predicate{iso_fun}{5}{+Operation, +SourceType, +TargetType, +Arg1, -ResultOfSourceType}
Transports a predicate of arity 2 F(+A,-B) to a domain where
an operation can be performed and brings back the result.

    \predicate{iso_fun}{6}{+Operation, +SourceType, +TargetType, +Arg1, +Arg2, -ResultOfSourceType}
Transports a predicate of arity 2 F(+A,+B,-C) to a domain where
an operation can be performed and brings back the result.
transports F(+A,+B,-C) 

    \predicate{iso_fun_}{6}{+Operation, +SourceType, +TargetType, +Arg1, -Res1, -Res2}
Transports a predicate of arity 2 F(+A,-B,-C) to a domain where
an operation can be performed and brings back the results.
transports F(+A,+B,-C) 

    \predicate{lazy_maplist}{3}{+F, +LazyXs, -LazyYs}
Applies a predicate to a lazy list resulting in anoter lazy list
Works with infinite list as input.

Lazy lists are not plain lists, as proven by applying maplist:
This loops!
\Sdirective{}\verb$lazy_nats(Ns)$,\verb$maplist(succ,Ns,Ms)$.

\predref{lazy_maplist}{3} fixes that.

    \predicate{lazy_maplist}{4}{+F, +LazyXs, LazyYs, -LazyYs}
like \predref{maplist}{4}, but working on (possibly infinite) lazy lists

    \predicate{split}{3}{+E, -E1, -E2}
\predref{split}{3} uses lazy lists to split a stream into two.
infelicity: original stream shifted by one position ...

    \predicate{lazy_conv}{3}{+As, +Bs, -Ps}
convolution of two finite or infinite lazy lists

    \predicate{convolution}{3}{+Gen1, +Gen2, -NewGen}
convolution of two finite or infinite lazy generators

    \predicate{eval_stream}{2}{+GeneratorExpression, -Generator}
evaluates a generator expression to ready to use
generator that combines their effects

    \infixop{in_}{-X}{+GeneratorExpression}
backtracks over elements of a generator expression
note that in_/2 is an xfx 800 operator, used as \arg{X} in_ Gen

    \predicate{ask_}{2}{GeneratorExpression, -Element}
produces next element after evaluating a gnerator expression

    \predicate{term_reader}{2}{+File, -TermGenerator}
creates a generator advancing on terms read from a file

    \predicate{fact}{2}{+N, -ResultGenerator}
factorial computation - use \predref{ask}{2} to extract value
used for testing

    \predicate{fibo}{1}{-Generator}
infinite Fibonacci stream for testing
\end{description}


\printindex
\end{document}
