\usepackage{graphicx}
\usepackage{mathptmx}
\usepackage{tikz-qtree}
\usepackage{pgfplots}
\usepackage{filecontents}
\usepackage{csvsimple}
\usepackage{amsfonts,amsmath}
%\usepackage{subfig} % creates problems - fails maketitile
\usepackage{url}
\usepackage{verbatim}
\usepackage{color}
%\usepackage[usenames,dvipsnames,svgnames,table]{xcolor}

\definecolor{lgray}{gray}{0.95}
\definecolor{lblue}{rgb}{0.90,0.90,1.00}
\definecolor{lyellow}{rgb}{1.00,1.00,0.70}

\usepackage{listings}
\lstloadlanguages{Prolog}
% parameters in this will be overriden by same for {code}
% so this will look the same - except that it will not be treated as code

%\newtheorem{prop}{Proposition}
%\newtheorem{conj}{Conjecture}
%\newtheorem{fact}{Fact}
%\newtheorem{df}{Definition}
%\newtheorem{lem}{Lemma}
\newtheorem{ex}{Example}
%\newtheorem{theo}{Theorem}

\newenvironment{codex}{\small\verbatim}{\endverbatim\normalsize}

\newcommand{\colorON}{\lstset{backgroundcolor=\color{lyellow}}}
\newcommand{\colorOFF}{\lstset{backgroundcolor=\color{white}}}


\let\codeh\comment 

\lstnewenvironment{code}
    {\lstset{}%
      \csname lst@SetFirstLabel\endcsname}
    {\csname lst@SaveFirstLabel\endcsname}
    \lstset{
      basicstyle=\small\ttfamily,
      % frame=single, 
      backgroundcolor=\color{lgray},
      flexiblecolumns=false,
      basewidth={0.5em,0.45em},        
    }

\newcommand{\BI}[0]{\begin{itemize}}
\newcommand{\EI}[0]{\end{itemize}}
\newcommand{\I}[0]{\item}
\newcommand{\BE}[0]{\begin{enumerate}}
\newcommand{\EE}[0]{\end{enumerate}}
\newcommand{\PP}[0]{ \vskip 0.6cm }
\newcommand{\BX}[0]{\begin{ex}}
\newcommand{\EX}[0]{\end{ex}}
\newcommand{\BV}[0]{\begin{verbatim}}
\newcommand{\EV}[0]{\end{verbatim}}

\newcommand{\BF}[0]{\begin{filecontents*}{data.csv}}
%\newcommand{\EF}[0]{\end{filecontents*}} NOT WORKING
  
\newcommand{\BQ}[0]{\color{blue}\begin{quote}}
\newcommand{\EQ}[0]{\end{quote}\color{black}}  
  
  
% \PIC{1 xlabel}{2 ylabel}{3 Title Label}{4 label}{
\newcommand{\PIC}[4]{
\begin{figure}[htbp]
\begin{center}
\begin{tikzpicture}
\begin{axis}[xlabel={#1},ylabel={#2}]
\addplot table [col sep=comma] {data.csv};
\end{axis}
\end{tikzpicture}
\end{center}
\caption{#3}
\label{#4}
\end{figure}
}  

  
% \LPIC{1 xlabel}{2 ylabel}{3 Title Label}{4 label}{
\newcommand{\LPIC}[4]{%\small
\begin{figure}[htbp]
\begin{center}
\begin{tikzpicture}
\begin{semilogyaxis}[xlabel={#1},ylabel={#2} (log. scale)]
\addplot table [col sep=comma] {data.csv};
\end{semilogyaxis}
\end{tikzpicture}
\end{center}
\caption{#3}
\label{#4}
\end{figure}
}  

% \PICS{1 xlabel}{2 ylabel}{3 legend1}{5 legend2}{5 Title Label}{6 label}{
\newcommand{\PICS}[6]{%\small
\begin{figure}[htbp]
\begin{center}
\begin{tikzpicture}
\begin{axis}[xlabel={#1},ylabel={#2},
		legend entries={#3,#4},
		legend style={
%			at={(1.03,0.5)},
      at={(0.50,1.03)},
			anchor=south
		}
]
\addplot table [x=a,y=b,col sep=comma] {data.csv};
\addplot table [x=a,y=c,col sep=comma] {data.csv};
\end{axis}
\end{tikzpicture}
\end{center}
\caption{#5}
\label{#6}
\end{figure}
}  

% \LPICS{1 xlabel}{2 ylabel}{3 legend1}{5 legend2}{5 Title Label}{6 label}{
\newcommand{\LPICS}[6]{%\small
\begin{figure}[htbp]
\begin{center}
\begin{tikzpicture}
\begin{semilogyaxis}[xlabel={#1},ylabel={#2} (log. scale),
		legend entries={#3,#4},
		legend style={
%			at={(1.03,0.5)},
      at={(0.50,1.03)},
			anchor=south
		}
]
\addplot table [x=a,y=b,col sep=comma] {data.csv};
\addplot table [x=a,y=c,col sep=comma] {data.csv};
\end{semilogyaxis}
\end{tikzpicture}
\end{center}
\caption{#5}
\label{#6}
\end{figure}
}  
   
    
\def \bscale1 {0.25}
\def \bscale {0.25}
\def \N {\mathbb{N}}
\def \T {\mathbb{T}}

% \FIG{Label}{Title}[scale]{file.suf}
\newcommand{\FIG}[4]{
\begin{figure}[htbp]
\centering
{\includegraphics[scale=#3]{../figs/#4}}
\caption{#2}
\label{#1}
\end{figure}
}

% \FIG{Label}{Title}[scale]{file.suf}
\newcommand{\CFIG}[4]{
\begin{figure}[htbp]
\centering
{\includegraphics[scale=#3,viewport=100 100 500 500,clip]{../figs/#4}}
\caption{#2}
\label{#1}
\end{figure}
}
% totalheight=0.4\textheight

% \HFIGS{Label}{Title}{Title1}{Title2}{PNG1}{PNG2}
\newcommand{\HFIGS}[6]{
\begin{figure}[htbp]
  \centering
  \subfloat[#3]{
    {\includegraphics[scale=0.60]{../figs/#5}}}   
  \subfloat[#4]{
    {\includegraphics[scale=0.60]{../figs/#6}}}
  \caption{#2}
  \label{#1}
\end{figure}
}

% \VFIGS{Label}{Title}{Title1}{Title2}{PNG1}{PNG2}
\newcommand{\VFIGS}[6]{
\begin{figure}[htbp]
  \begin{center}
    {\includegraphics[scale=0.40]{../figs/#5}}
    {\includegraphics[scale=0.40]{../figs/#6}}
  \caption{#2: {\em #3} and {\em #4}}
  \label{#1}
  \end{center}
\end{figure}
}


% \VSFIGS{Label}{Title}{Title1}{Title2}{PNG1}{PNG2}{Scale}
\newcommand{\VSFIGS}[7]{
\begin{figure}[htbp]
  \begin{center}
    {\includegraphics[scale=#7]{../figs/#5}}
    {\includegraphics[scale=#7]{../figs/#6}}
  \caption{#2 #3 #4}
  \label{#1}
  \end{center}
\end{figure}
}

% CSV{Title}{Label}
\newcommand{\CSV}[2]{
\begin{figure}[htbp]
\begin{center}
\csvautotabular{data.csv}
\caption{#1}
\label{#2}
\end{center}
\end{figure}
}


